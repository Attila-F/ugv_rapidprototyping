\section{Integráció}

\subsection{A generált kód felépítése}

Amennyiben compact code placement beállítással generáltuk le a fileokat, csupán 4 számunkra hasznos file keletkezik:

\begin{itemize}

    \item "subsystem neve".c
    \item "subsystem neve".h
    \item rtwtypes.h
    \item ert\_main.c

\end{itemize}

A továbbiakban feltételezzük, hogy a generált subsystem neve "Controller" volt, az egyszerűbb olvashatóságért.

\begin{figure}[!ht]
    \centering
    \includegraphics[width=0.6\linewidth]{img/rtw}
    \caption{A generált fileok kapcsolata}
    \label{fig:rtw}
\end{figure}

A \textbf{Controller.c} és \textbf{Controller.h} fileokban van definiálva a rendszer működése. A kódból hozzávetőlegesen kiolvasható a modell működése. A \textbf{rtwtypes.h} fileban kerülnek definiálásra a rendszer által használt típusok, az \textbf{ert\_main.c} pedig bemutatja a használatukat.

\subsection{Simulink modell kezelése C kódban}

A generált kód tökéletes megfelelője a Simulinkben futtatott modellnek. A generált fileok egy objektumot definiálnak, saját típusokkal és tagfüggvényekkel.

A rendszer inicializálását a Controller\_initialize() függvénnyel tehetjük meg. Ez a modell futásának elídításának felel meg, beállítja a 0 időpillanatot a rendszerben, a kimenetek felveszik az alapértelmezett értéküket.
A rendszerrel csupán a be és kimeneti interfészen keresztül kommunikálhatunk, valamint hatást gyakorolhatunk rá a \verb!Step! függvény meghívási idejének változtatásával (ezt nem részletezzük, és ne is nagyon erőltessük).
A bemeneti interfészt a Controller\_U struktúra implementálja, ennek a változónak a tagjai felelnek meg egy-egy bemeneti jelnek.\footnote{Többdimenziós jel esetén tömbként történik az átadás. 2 vagy több dimenzió esetén ne feledjük, hogy a MATLAB \textbf{oszlopfolytonosan} kezeli a tömböket.}

Miután beállítottuk a bemeneteket, a Controller\_step() függvény meghívásával \textbf{léptethetjük} a modellt. A lépés megegyezik a generált rendszer \textbf{1 mintavételi idejű} lépésével. A Step függvény lefutása közben frissülnek a rendszer belső állapotai, valamint Controller\_Y-ban a kimenetek. A kimenetet hasonlóléppen állítja elő a rendszer, Controller\_Y struktúrában tárolva.

A bemenetek megadása, léptetés, kimenet kiolvasása az az elemi lépéssorozat, melyet rendszeresen \textbf{időzítve} végrehajtunk. Ha a modellnek időtől függő belső állapotai is vannak\footnote{Ez minden esetben igaz, P szabályzónál bonyolultabb irányítás esetében}, \textbf{pontos időzítéssel} kell biztosítanunk a periodikus meghívást. Ha a rendszer túlnyomó része MATLAB-ban készült, ez csupán timer időzítővel is biztosítható, ám ha más feladatokat is el kell látnia a rendszernek, valószínűleg egy Hard Real-Time operációs rendszerre is szükségünk lesz.

Rögtön felhívnám a figyelmet, ha eddig nem vált volna nyilvánvalóvá, hogy a generált forrásfileok közül \textbf{csak} az ert\_main.c-t szabad módosítani. Ha úgy érezzük, hogy a többi generált kódba kell kézzel belenyúlni, akkor valamit elrontottunk. Továbbá meg lehet találni a megfelelő megfeleltetéseket a generált kód és a simulink modell között, de ez többnyire felesleges és \textbf{megbízhatatlan}. A legjobb fekete dobozként tekinteni a rendszerre, ha pedig valamilyen belső változóra szükség van debuggoláshoz, vegyük a fáradságot és vezessük ki kimenetre\footnote{A Goto/From simulink blokkpárral ezt elegánsan megtehetjük, a vezetékek összekuszálása nélkül}.

\subsection{Példarendszer integrációja periodikus meghívással}

A példarendszerünk bemenete a vonal pozíciója, kimenete pedig a kívánt kormányszög. A főprogramunkból a következőképpen tudjuk meghívni a modellt:

\lstset{escapechar=@,style=customc}
\begin{lstlisting}[frame=single]
	/* Pass inputs */
	Controller_U.Position = line_position;

	/* Update model */
	Controller_step();

	/* Receive outputs */
	servo_position = Controller_Y.Servo;
\end{lstlisting}

Ezt a kódot kell egy olyan függvénybe beletennünk, melynek tudjuk biztosítani a periodikus meghívását.

A vonal pozíciót feldolgozhatjuk C-ben is akár, de minek, ha úgyis köré építünk egy MATLAB rendszert? Ehhez pointerrel tudjuk átadni az adattömböt, Simulinkben pedig vektor bemenetet kell beállítanunk.